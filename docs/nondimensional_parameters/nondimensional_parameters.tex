\documentclass[10pt]{article}
\usepackage{amsmath,amssymb}
\usepackage[margin=1.5in]{geometry}
\usepackage[colorlinks]{hyperref}
\usepackage{mathtools}

\usepackage{fancyhdr}
\pagestyle{fancy}
\fancyhead{}
\fancyfoot{}
\fancyhead[CO]{\textsc{\small Nondimensional Parameters for Unsteady Airfoils}}
\fancyfoot[L]{\href{http://mikef.org}{\emph{\small Michael J. Fairchild}}}
\fancyfoot[R]{\emph{\small Princeton University}}

\DeclarePairedDelimiter\abs{\lvert}{\rvert}
\newcommand\defn[1]{\emph{#1}}
\def\eg{e.g.~}
\def\ie{i.e.~}

\begin{document}
The \defn{convective velocity} is the velocity of the onset flow, whose speed we denote $U$.  If $c$ is a characteristic length (typically the airfoil chord), the \defn{convective time} $\tau$ is the time required for the onset flow to travel a distance equal to the characteristic length $c$, \ie \[\tau=c/U.\]  

If the airfoil is oscillating with \defn{fundamental frequency} $f$ (Hz) (and hence with period $T=1/f$), we define the \defn{reduced frequency} $k$ as the number of oscillations in a duration of one convective time, \ie \[k=f\tau.\]  Note that if $T=1/f$ is the period of oscillation, then $k=\tau/T=fc/U$. The definition $k=f\tau$ is a natural one, but other conventions are in use.  In particular, some authors define a reduced frequency $k'$ by \[k'=\frac{\omega c}{2U}=\pi k,\] where $\omega=2\pi f$ is the angular frequency (rad/sec) of oscillation.

If the motion (\eg of the airfoil's trailing edge) has amplitude $a$, the \defn{Strouhal number} is \[St = \frac{f\,a}{U}.\]

We have the following relationships between the definitions: \[\boxed{\frac{\tau}{T}=f\tau=2\pi\omega\tau=k=\frac{k'}{\pi}=\frac{f\,c}{U}=St\cdot\frac{c}{a}}\]

For multi-frequency motions, we have found it useful to introduce the \defn{generalized Strouhal number} \[St_* := \frac{\frac{1}{T}\int_0^T\abs{\dot y}}{U}\,dt,\] where $T$ is the period of the fundamental mode, and $\dot y$ is the heave velocity.

\end{document}